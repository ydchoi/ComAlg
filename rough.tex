\documentclass[11pt]{article}

\usepackage{amsmath, amssymb, amsthm}
\usepackage{tikz}

\theoremstyle{plain}
\newtheorem{thm}{Theorem}[section]
\newtheorem*{thm*}{Theorem}
\newtheorem{prop}[thm]{Proposition}
\newtheorem{lem}[thm]{Lemma}
\newtheorem*{lem*}{Lemma}
\newtheorem{dfn}[thm]{Definition}
\newtheorem{cor}[thm]{Corollary}
\newtheorem{claim}[thm]{Claim}
\newtheorem{conj}[thm]{Conjecture}
\newtheorem{ques}[thm]{Question}
\newtheorem*{rem}{Remark}


\oddsidemargin  0pt
\evensidemargin 0pt
\marginparwidth 40pt
\marginparsep 10pt
\topmargin 0pt
\headsep 10pt
\textheight 8.2in
\textwidth 6.4in
\renewcommand{\baselinestretch}{1.1}

\newcommand{\codeg}{\text{codeg}}
\newcommand{\BBE}{\mathbb{E}}
\newcommand{\BFP}{\mathbf{P}}
\usepackage{amsmath}
\usepackage{amsthm}
\usepackage{amssymb}
\usepackage{mathtools}
\usepackage{hyperref}
\usepackage{url}





\usepackage{graphicx}
\usepackage{caption}
\usepackage{subcaption}

\def\eQb#1\eQe{\begin{eqnarray*}#1\end{eqnarray*}}
\def\eQnb#1\eQne{\begin{eqnarray}#1\end{eqnarray}}
\providecommand{\e}[1]{\ensuremath{\times 10^{#1}}}
\providecommand{\pb}[0]{\pagebreak}
\DeclarePairedDelimiter\ceil{\lceil}{\rceil}
\DeclarePairedDelimiter\floor{\lfloor}{\rfloor}

\newcommand{\E}{\mathrm{E}}
\newcommand{\Var}{\mathrm{Var}}
\newcommand{\Cov}{\mathrm{Cov}}

\def\Qb#1\Qe{\begin{question}#1\end{question}}
\def\Sb#1\Se{\begin{solution}#1\end{solution}}


\newtheoremstyle{quest}{\topsep}{\topsep}{}{}{\bfseries}{}{ }{\thmname{#1}\thmnote{ #3}.}
\theoremstyle{quest}
\newtheorem*{definition}{Definition}
\newtheorem*{theorem}{Theorem}
\newtheorem*{lemma}{Lemma}
\newtheorem*{question}{Question}
\newtheorem*{preposition}{Preposition}
\newtheorem*{exercise}{Exercise}
\newtheorem*{challengeproblem}{Challenge Problem}
\newtheorem*{solution}{Solution}
\newtheorem*{remark}{Remark}
\usepackage{verbatimbox}
\usepackage{listings}
\usepackage{mathrsfs}
\date{}
\title{\vspace{-0.7cm}
Atiyah Commutative Algebra: Problems}

\author{
Youngduck Choi 
\thanks{Department of Mathematics, Courant Institute of Mathematical Sciences, 
yc1104@nyu.edu; If you find an error and want to share with me, 
you can reach me via email.
}}

\begin{document}

\maketitle

\begin{abstract}
This work contains solutions to some
exercises from Atiyah's Commutative Algebra
text.
\end{abstract}


\section{Chapter 2: Modules} \label{sec:Modules}

\begin{question}[2.1]
\hfill
\begin{figure}[h!]
  \centering
    \includegraphics[width=0.7\textwidth]{d-2-1.png}
\end{figure}
\end{question}
\begin{solution} \hfill \\
As $m,n$ are coprime, by Bezout, there exists $a,b \in \mathbb{Z}$ such that
\eQb
am + bn = 1.
\eQe
Therefore, 
\eQb
x \otimes y &=& am(x \otimes y) + bn(x \otimes y) = a (mx \otimes y) + 
b(x \otimes ny) = a(0 \otimes y ) + b(x \otimes 0) = 0 
\eQe
for any $x \otimes y \in \mathbb{Z}/n\mathbb{Z} 
\otimes_{\mathbb{Z}} \mathbb{Z}/m\mathbb{Z}$. 
\hfill $\qed$

\end{solution}

\bigskip

\begin{question}[2.4]
\hfill
\begin{figure}[h!]
  \centering
    \includegraphics[width=0.7\textwidth]{d-2-4.png}
\end{figure}
\end{question}
\begin{solution} \hfill \\
Let $M = \bigoplus_{i \in I} M_i$. Then, $M$ is exact iff for any exact sequence
of $A-$modules
\eQb
0 \to N' \to N \to N'' \to 0 
\eQe 
the tensored sequence 
\eQb
0 \to N' \otimes M \to N \otimes M \to N'' \otimes M \to 0 
\eQe 
is exact. Observe that the above tensored sequence can be written as
\eQb
0 \to \bigoplus_{i \in I} N' \otimes M_i \to 
\bigoplus_{i \in I} N \otimes M_i \to
\bigoplus_{i \in I} N'' \otimes M_i \to 0. 
\eQe
Hence $M$ is exact iff for any $i \in I$ and for any exact sequence of $A-$modules
\eQb
0 \to N' \to N \to N'' \to 0 
\eQe
the tensored sequence
\eQb
0 \to N' \otimes M_i  \to N \otimes M_i \to N'' \otimes M_i \to 0, 
\eQe
but the later is equivalent to $M_i$ being flat for all $i \in I$, so we are done.
\hfill $\qed$

\end{solution}

\bigskip

\begin{question}[2.5]
\hfill
\begin{figure}[h!]
  \centering
    \includegraphics[width=0.7\textwidth]{d-2-5.png}
\end{figure}
\end{question}
\begin{solution} \hfill \\

\end{solution}

\newpage

\section{Chapter 1: Rings and Ideals} \label{sec:RaI}

\begin{question}[1.1]
\hfill
\begin{figure}[h!]
  \centering
    \includegraphics[width=0.7\textwidth]{d-1-1.png}
\end{figure}
\end{question}
\begin{solution} \hfill \\


\end{solution}
\begin{question}[1.2]
\hfill
\begin{figure}[h!]
  \centering
    \includegraphics[width=0.7\textwidth]{d-1-2.png}
\end{figure}
\end{question}
\begin{solution} \hfill \\


\end{solution}
\begin{question}[1.3]
\hfill
\begin{figure}[h!]
  \centering
    \includegraphics[width=0.7\textwidth]{d-1-3.png}
\end{figure}
\end{question}
\begin{solution} \hfill \\


\end{solution}
\begin{question}[1.4]
\hfill
\begin{figure}[h!]
  \centering
    \includegraphics[width=0.7\textwidth]{d-1-4.png}
\end{figure}
\end{question}
\begin{solution} \hfill \\
Let $\mathfrak{N}$, and $\mathfrak{R}$ be nilradical, and Jacobson radical 
respectively. As any primal ideal is a maximal idea,
\eQb
\mathfrak{N} &=& \bigcap_{\{\mathfrak{p} \> : \> \text{a primal ideal of} 
\> A\}} \mathfrak{p} \\ 
&\subset& 
\bigcap_{\{ \mathfrak{p} \> : \> \text{a maximal ideal of } \> A\}} \mathfrak{p}
= \mathfrak{R}.
\eQe
Hence, it now suffices to show that $\mathfrak{R} \subset \mathfrak{N}$.
Let $f = a_0 + a_1x + ... + a_n x^n \in \mathfrak{R}$. Then, by preposition 1.9,
\eQb
1 + xf = 1 + a_0x + ... a_n x^n{n+1} \>\>\> \text{is a unit in} \>\>\> A[x].
\eQe 
Now, by exercise 1.2-i, $a_0 , ... , a_n$ are nilpotent, and by 1-2-ii,
$f \in \mathfrak{N}$. Therefore, $\mathfrak{N} = \mathfrak{R}$. \hfill $\qed$


\end{solution}

\bigskip

\begin{question}[1.5]
\hfill
\begin{figure}[h!]
  \centering
    \includegraphics[width=0.7\textwidth]{d-1-7.png}
\end{figure}
\end{question}
\begin{solution} \hfill \\
Let $\mathfrak{p}$ be a prime ideal of $A$. 

\end{solution}

\newpage

\begin{question}[1.15]
\hfill
\begin{figure}[h!]
  \centering
    \includegraphics[width=0.7\textwidth]{d-1-15.png}
\end{figure}
\end{question}
\begin{solution} \hfill \\
\textbf{(i)} Any prime ideal containing $r(\mathfrak{a})$ contains $\mathfrak{a}$,
so $V(r(\mathfrak{a})) \subset V(\mathfrak{a})$. Now, if $\mathfrak{p} \in 
V(\mathfrak{a})$, then by closure of multiplication of $\mathfrak{p}$, $r(\mathfrak{a}) 
\subset \mathfrak{p}$. Hence, $V(\mathfrak{a}) \subset V(r(\mathfrak{a}))$, and
$V(\mathfrak{a}) = V(r(\mathfrak{a}))$. Now, any prime ideal containing $E$,
as $\mathfrak{a}$ is generated by $E$, contains $E$, therefore $V(\mathfrak{a}) 
\subset V(E)$. Now, let $\mathfrak{p} \in V(E)$. Then, as by definition, $\mathfrak{a}$
is the smallest ideal that contains $E$, so $\mathfrak{a} \subset \mathfrak{p}$, 
and therefore $V(E) = V(\mathfrak{a}) = V(r(\mathfrak{a}))$. This shows that
\eQb
E \subset \mathfrak{p} \iff \mathfrak{a} \subset \mathfrak{p} \iff 
r(\mathfrak{a}) \subset \mathfrak{p}
\eQe
for any prime ideal $\mathfrak{p}$ of $A$. \\  

\smallskip 

\noindent \textbf{(ii)} It is clear that $V(0) \subset X$. Any prime ideal $\mathfrak{p}$
of $A$ contains $0$, so $V(0) = X$. And as there is no prime ideal that contains
$1$, $V(1) = \emptyset$.  \\


\smallskip

\noindent \textbf{(iii)} Let $\mathfrak{p} \in V(\bigcup_{i \in I} E_i)$. Then,
$E_i \subset \mathfrak{p}$ for all $i \in I$. Therefore,
$\mathfrak{p} \in V(E_i)$ for all $i \in I$, so $\mathfrak{p} \in \bigcap_{i \in I}
V(E_i)$. This shows $V(\bigcup_{i \in I} E_i ) \subset \bigcap_{i \in I} 
V(E_i)$. Now, suppose $\mathfrak{p} \in \bigcap_{i \in I} V(E_i)$. Then,  
$E_i \subset \mathfrak{p}$ for all $i \in I$, so $\bigcup_{i \in I} V(E_i) 
\subset \mathfrak{p}$. Therefore, $p \in V(\bigcup_{i \in I} E_i)$, so we are done. \\

\smallskip

\noindent We can also see
\eQb
\mathfrak{p} \in \bigcap_{i \in I} V(E_i) \iff p \in V(E_i) \>\>\> \forall i \in I
\iff E_i \subset \mathfrak{p} \>\>\> \forall i \in I \\
\iff \bigcup_{i \in I} E_i \subset \mathfrak{p} \iff \mathfrak{p} \in 
V(\bigcup_{i \in I} E_i) 
\eQe 
for any prime ideal $\mathfrak{p}$ in $A$. 

\smallskip

\noindent \textbf{(iv)} For any prime ideal $\mathfrak{p}$ of $A$,
\eQnb
\mathfrak{p} \in V(\mathfrak{a} \cap \mathfrak{b}) &\implies& 
\mathfrak{a} \cap \mathfrak{b} \subset \mathfrak{p} \implies
\mathfrak{a}\mathfrak{b} \subset \mathfrak{a} \cap \mathfrak{b} \subset \mathfrak{p} 
\nonumber \\
&\implies& \mathfrak{p} \in V(\mathfrak{a}\mathfrak{b}) 
\implies \mathfrak{a} \subset \mathfrak{p} \>\>\> \text{or} \>\>\>
\mathfrak{b} \subset \mathfrak{p} \label{eq:1-15-1} \\
&\implies& \mathfrak{p} \in V(\mathfrak{a}) \>\>\> \text{and} \>\>\> \mathfrak{p} \in
V(\mathfrak{b})
\implies \mathfrak{p} \in V(\mathfrak{a}) \cup V(\mathfrak{b}) \nonumber \\
&\implies& \mathfrak{a} \subset \mathfrak{p} \>\>\> \text{or} 
\mathfrak{b} \subset \mathfrak{p} 
\implies \mathfrak{a} \cap \mathfrak{b} \subset \mathfrak{p} 
\implies \mathfrak{p} \in V(\mathfrak{a} \cap \mathfrak{b}) \nonumber  
\eQne
where~\eqref{eq:1-15-1} holds by primality of $\mathfrak{p}$. Therefore,
\eQb
V(\mathfrak{a} \cap \mathfrak{b}) = V(\mathfrak{a}\mathfrak{b}) = 
V(\mathfrak{a}) V(\mathfrak{b}),
\eQe
as required. \hfill $\qed$

\end{solution}

\bigskip

\begin{question}[1.16]
\hfill
\begin{figure}[h!]
  \centering
    \includegraphics[width=0.7\textwidth]{d-1-16.png}
\end{figure}
\end{question}
\begin{solution} \hfill \\


\end{solution}

\newpage

\begin{question}[1.17]
\hfill
\begin{figure}[h!]
  \centering
    \includegraphics[width=0.7\textwidth]{d-1-17.png}
\end{figure}
\end{question}
\begin{solution} \hfill \\
We claim that $\mathscr{B} = 
\{X_f = X \setminus V(f)\}_{f \in A}$ is a basis of the Zariski 
topology on $X$. Let $O$ be an open subset in the Zariski topology, so
$O = X \setminus V(E)$ for some $E \in 2^{A}$. Then,
\eQnb
O &=& X \setminus V(E) = X \setminus V(\bigcup_{f \in E} \{f\}) \nonumber \\
&=& X \setminus \bigcap_{f \in E} V(\{f\}) = X \setminus \bigcap_{f \in E} X_f 
\label{eq:1-17-1}\\
&=& \bigcup_{f \in E} X \setminus X_f \label{eq:1-17-2} 
\eQne
where~\eqref{eq:1-17-1} holds by 1-15-(iii) and~\eqref{eq:1-17-2} holds by 
DeMorgan's law. Therefore, any open $O$ in the Zariski topology is a union of elements
in $\mathscr{B}$, so $\mathscr{B}$ is a basis for the Zariski topology. \\

\smallskip 

\noindent \textbf{(i)}
For any prime ideal $\mathfrak{p}$ of $A$,
\eQnb
\mathfrak{p} \in X_f \cap X_g &\iff& \mathfrak{p} \in X_f \>\>\> \text{and}
\>\>\> \mathfrak{p} \in X_g 
\iff \mathfrak{p} \not\in V(f) \>\>\> \text{and} \>\>\> \mathfrak{p} \not\in V(g) 
\nonumber \\
&\iff& f \not\in \mathfrak{p} \>\>\> \text{and} \>\>\> g \not\in \mathfrak{p} 
\iff fg \not\in \mathfrak{p} \label{eq:1-17-3} \\ 
&\iff& \mathfrak{p} \not\in V(fg) \iff 
\mathfrak{p} \in X_{fg}  \nonumber
\eQne
where~\eqref{eq:1-17-3} holds by primality of $\mathfrak{p}$. Therefore,
$X_{f} \cap X_{g} = X_{fg}$.

\smallskip

\noindent \textbf{(ii)} Observe that 
\eQnb
X_f = \emptyset &\iff& X \subset V(f) \iff \>\>\> \forall \>\>\> \text{prime ideal} \>
\mathfrak{p} \> \text{of} \> A \>\>\> f \in \mathfrak{p} \nonumber \\
&\iff& f \in \bigcap_{\mathfrak{p} : \> \text{prime ideal of } A} \mathfrak{p} 
\iff f \>\>\> \text{belongs to the nilradical of} \>\>\> f \label{eq:1-17-4} \\ 
&\iff& f \>\>\>  \text{is nilpotent.} \nonumber 
\eQne
where~\eqref{eq:1-17-4} holds by preposition 1-8. 

\smallskip

\noindent \textbf{(iii)} 

\end{solution}

\bigskip

\begin{question}[1.18]
\hfill
\begin{figure}[h!]
  \centering
    \includegraphics[width=0.7\textwidth]{d-1-18.png}
\end{figure}
\end{question}
\begin{solution} \hfill \\

\end{solution}

\bigskip

\begin{question}[1.19]
\hfill
\begin{figure}[h!]
  \centering
    \includegraphics[width=0.7\textwidth]{d-1-19.png}
\end{figure}
\end{question}
\begin{solution} \hfill \\

\end{solution}

\bigskip

\begin{question}[1.21]
\hfill
\begin{figure}[h!]
  \centering
    \includegraphics[width=0.7\textwidth]{d-1-21.png}
\end{figure}
\end{question}
\begin{solution} \hfill \\

\end{solution}


\section{Chapter 6: Chain Conditions} \label{sec:CC}

\begin{question}[6.2]
\hfill
\begin{figure}[h!]
  \centering
    \includegraphics[width=0.7\textwidth]{d-6-2.png}
\end{figure}
\end{question}
\begin{solution} \hfill \\
Suppose $M$ is not Noetherian. Then, there exists a non finitely-generated submodule
$N$ of $M$. Choose $x_1$ from $N$. Then, since $N$ is non finitely-generated,
$N - \{ x_1 \} \neq \emptyset$, so we can choose $x_2 \neq x_1$ from $N$. Repeat
this process inductively, then we have a nonempty set of finitely generated
submodules of $M$ 
\eQb
\{ (x_1), (x_1, x_2), (x_1, x_2, x_3) ... \}
\eQe
such that it does not have a maximal element. Therefore, by contradiction,
we have that $M$ must be Noetherian. \hfill $\qed$  

\end{solution}

\bigskip

\section{Chapter 3: Rings and Modules of Fractions} \label{sec:RMF}
\begin{question}[3.1]
\hfill
\begin{figure}[h!]
  \centering
    \includegraphics[width=0.7\textwidth]{d-3-1.png}
\end{figure}
\end{question}
\begin{solution} \hfill \\
Suppose there exists $s_0 \in S$ such that $s_0 M = 0$. Then, for any 
$(a,s), (b,t) \in A \times S$,  
\eQb
(at - bs) s_0 = 0
\eQe
so $(a,s) \sim (b,t)$, and hence $S = 0$. 
Conversely, suppose $S^{-1}M = 0$. As $M$ is finitely generated,
there exists $\{x_1 , ..., x_m\}$ for some $m$ that generates $M$. As $S^{-1}M = 0$,
we can choose $\{s_1, ..., s_m\}$ such that
\eQb
s_i x_i = 0
\eQe 
for all $1 \leq i \leq m$. Consider $s^* = s_1...s_m$. Then,
\eQb
s^* x &=& s_1...s_m x = s_1 ... s_m (\sum_{i=1}^{m} a_i x_i ) \\
&=& \sum_{i=1}^{m} s_1 ... s_m a_i x_i =  0  
\eQe 
for some $a_1,...a_m \in A$, for all $x \in M$. Therefore, $s^*M = 0$.
\hfill $\qed$

\end{solution}

\bigskip

\begin{question}[3.2]
\hfill
\begin{figure}[h!]
  \centering
    \includegraphics[width=0.7\textwidth]{d-3-2.png}
\end{figure}
\end{question}
\begin{solution} \hfill \\
If $1+a, 1+a' \in 1+ \textbf{a}$, then
\eQb
(1+a)(1+a') &=& 1 a + a' + aa' \in 1 + \textbf{a}. 
\eQe 
Therfore $1+\textbf{a}$ is a multiplicatively closed set in $A$. Now,
for any $\in 1 + \textbf{a}$, 
\eQb
1 - 1
\eQe

\end{solution}

\newpage


\begin{question}[3.5]
\hfill
\begin{figure}[h!]
  \centering
    \includegraphics[width=0.7\textwidth]{d-3-5.png}
\end{figure}
\end{question}
\begin{solution} \hfill \\
Let $\mathfrak{R}$ be a nilradical of $A$. By Corollary $3.12$, 
$(\mathfrak{R})_{\mathfrak{p}}$ is a nilradical of $A_{\mathfrak{p}}$, which is 0 by
assumption, for any prime ideal $\mathfrak{p}$. Therefore, by proposition 3.8,
$\mathfrak{R} = 0$.

\smallskip

 
\end{solution}

\newpage

\begin{question}[3.12]
\hfill
\begin{figure}[h!]
  \centering
    \includegraphics[width=0.7\textwidth]{d-3-12.png}
\end{figure}
\end{question}
\begin{solution} \hfill \\
We first show that $T(M)$ is a A-sub-module of $M$. Let $x,y \in T(M)$, and $r \in A$.
Then, there exists $a,b \in A$, such that $a,b \neq 0$, $ax = 0$ and $by = 0$. 
Then, as $A$ is an integral domain, $ab \neq 0$ and
$ab(x+y) = abx + aby = bax + aby = 0$, so $x + y \in T(M)$. Also, $arx = rax = 0$,
so $rx \in T(M)$ and $T(M)$ is a $A-$sub-module of $M$. \\ 

\smallskip

\noindent 
\textbf{(i)} To show that $M/T(M)$ is torsion-free, it suffices to show that
$T(M/T(M)) = 0$.  \\

\smallskip

\noindent
\textbf{(ii)} Let $y \in f(T(M))$. Then, there exists $x \in T(M)$ such that
$f(x) = y$, and $0 \neq a \in A$ with $ax = 0$. Therefore, $ay = af(x) = f(ax) = 0$,
as $f$ is a module homomorphism. Therefore, $f(T(M)) \subset T(N)$. \\ 

\smallskip

\noindent \textbf{(iii)}

\smallskip


\end{solution}

\bigskip

\begin{question}[3.13]
\hfill
\begin{figure}[h!]
  \centering
    \includegraphics[width=0.7\textwidth]{d-3-13.png}
\end{figure}
\end{question}
\begin{solution} \hfill \\
\end{solution}



\end{document}
